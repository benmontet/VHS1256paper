\documentclass[twocolumn]{aastex63}

\bibliographystyle{aasjournal}
\usepackage{subfigure}
\usepackage{url}
\usepackage{hyperref}
%\usepackage{datetime}
\usepackage{longtable}
\usepackage{natbib}
\usepackage{amsmath}
\usepackage{listings}
\usepackage[normalem]{ulem}
\usepackage{bm}
\usepackage{comment}
% additions for ease of tabling
\usepackage{array}
\newcolumntype{P}[1]{>{\centering\arraybackslash}p{#1}}
\newcolumntype{M}[1]{>{\centering\arraybackslash}m{#1}}
% colors
%\usepackage[usenames, dvipsnames]{color}

%\usepackage{pdflscape}
\usepackage{xcolor, fontawesome}
\definecolor{twitterblue}{RGB}{64,153,255}
\newcommand{\twitter}[1]{\href{https://twitter.com/#1 }{\textcolor{twitterblue}{\faTwitter}\,\tt \textcolor{twitterblue}{@#1}}}

\definecolor{Code}{rgb}{0,0,0}
\definecolor{Decorators}{rgb}{0.5,0.5,0.5}
\definecolor{Numbers}{rgb}{0.5,0,0}
\definecolor{MatchingBrackets}{rgb}{0.25,0.5,0.5}
\definecolor{Keywords}{rgb}{1,0,0}
\definecolor{self}{rgb}{0,0,0}
\definecolor{Strings}{rgb}{0,0.63,0}
\definecolor{Comments}{rgb}{0,0.63,1}
\definecolor{Backquotes}{rgb}{0,0,0}
\definecolor{Classname}{rgb}{0,0,0}
\definecolor{FunctionName}{rgb}{0,0,0}
\definecolor{Operators}{rgb}{0,0,0}
\definecolor{Background}{rgb}{0.98,0.98,0.98}
\definecolor{Booleans}{rgb}{0.572,0,0.572}
\definecolor{BuiltinFunction}{rgb}{0.572,0,0.572}
\definecolor{BuiltinConstant}{rgb}{0.572,0,0.572}
\definecolor{Asterisk}{rgb}{0.670,0,1}

\lstdefinelanguage{Python}{
        numbers=left,
        numberstyle=\footnotesize,
        numbersep=7pt,
        xleftmargin=1.26em,
        framextopmargin=2em,
        framexbottommargin=2em,
        showspaces=false,
        showtabs=false,
        showstringspaces=false,
        frame=l,
        tabsize=4,
        stepnumber=1,
    % Basic
    basicstyle=\small\ttfamily,
        backgroundcolor=\color{Background},
%        breaklines=True,
%        postbreak=\mbox{\textcolor{red}{$\hookrightarrow$}\space},
    % Comments
%    commentstyle=\color{green}\ttfamily,
    % Strings
%        stringstyle=\ttfamily\color{Strings},
%        morecomment=[s][\color{Strings}]{'}{'}, 
%        stringstyle=\ttfamily\color{Comments},
%        morecomment=[s][\color{Comments}]{\#}{\#},             
    % Keywords
    stringstyle=\ttfamily\color{Strings},
    morekeywords={import,from,class,def,while,if,in,elif,else,not,or,print,break,continue,return,access,as,except,exec,finally,global,import,lambda,pass,print,raise,try,assert},
        keywordstyle={\color{Keywords}\bfseries}, 
    %    morekeywords={[2]True,False,None},
    %    keywordstyle={[2]\color{BuiltinConstant}\slshape},
    otherkeywords={[2]*},
    keywordstyle={[2]\color{Asterisk}},
%    emph={self},
%    emphstyle={\color{self}\slshape}    
}

\usepackage{color}

\newcommand{\ron}{\color{red}} 
\newcommand{\bon}{\color{blue}} 
\newcommand{\gon}{\color{green}} 
\newcommand{\coff}{\color{black}\,}
\newcommand{\shrug}{\texttt{\raisebox{0.75em}{\char`\_}\char`\\\char`\_\kern-0.5ex(\kern-0.25ex\raisebox{0.25ex}{\rotatebox{45}{\raisebox{-.75ex}"\kern-1.5ex\rotatebox{-90})}}\kern-0.5ex)\kern-0.5ex\char`\_/\raisebox{0.75em}{\char`\_}}}


\newcommand{\rprs}{{$R_p/R_{\star}$}}

\newcommand{\eg}{{\it e.g.}}
\newcommand{\ie}{{\it i.e.}}
\newcommand{\kep}{{\it Kepler}}
\newcommand{\kt}{{\it K2}}
\newcommand{\tess}{{\it TESS}}
\newcommand{\ffis}{Full-Frame Images}
\newcommand{\fulltess}{{\it Transiting Exoplanet Survey Satellite}}
\newcommand{\Gaia}{{\it Gaia}}
\newcommand{\spitz}{{\it Spitzer}}
\newcommand{\vsini}{{$v \sin i$}}
\newcommand{\teff}{$T_{ eff}$}
\newcommand{\kms}{{km\,s$^{-1}$}}
\newcommand{\gcc}{{g\,cm$^{-3}$}}
\newcommand{\rstar}{{$R_\star$}}
\newcommand{\rhostar}{{$\rho_\star$}}
\newcommand{\mearth}{{M$_\oplus$}}
\newcommand{\rearth}{{R$_\oplus$}}
\newcommand{\rsun}{{R$_\odot$}}
\newcommand{\msun}{{M$_\odot$}}
\newcommand{\mjup}{{M$_\textrm{Jup}$}}
\newcommand{\rjup}{{R$_\textrm{Jup}$}}

\newcommand{\thissysshort}{{VHS J1256}}
\newcommand{\thisstar}{\thissystem AB}
\newcommand{\thissystem}{{\thissysshort –1257}}

\newcommand{\eleanor}{\texttt{eleanor}}

\newcommand{\mstar}{{$M_\star$}}
\newcommand{\logg}{{log(g)}}
\newcommand{\mh}{{[M/H]}~}
\newcommand{\feh}{{[Fe/H]}~}
%\newcommand{\h2ok2}{{$ H_2O-K2$}}

\newcommand{\todo}[3]{{\color{#2} \emph{#1} TO DO: #3}}
\newcommand{\btmtodo}[1]{\todo{BEN}{blue}{#1}}
\newcommand{\anytodo}[1]{\todo{ANYONE}{green}{#1}}

\newcommand{\comm}[1]{{\color{cyan}{#1}}}

\newcommand{\Sph}{{$S_{\textrm{ph}}$}}
\newcommand{\RHK}{{$R'_{\textrm{HK}}$}}

\newcommand{\unsw}{School of Physics, University of New South Wales, Sydney, NSW 2052, Australia}


\newcommand{\ut}{UT Austin, USA}


%\submitted{for May 2, 2016}

\begin{document}
\title{VHS 1256 Paper}

\shorttitle{VHS 1256 Paper To be Named} 
\shortauthors{Montet et al.}


\author[0000-0001-7516-8308]{Benjamin~T.~Montet}
\affiliation{\unsw}


\author{Yifan Zhou}
\affiliation{\ut}

\author{Marta L. Bryan}
\affiliation{Department of Astronomy, 501 Campbell Hall, University of California Berkeley, Berkeley, CA 94720-3411, USA}

\author{Brendan P. Bowler}
\affiliation{\ut}


\correspondingauthor{Benjamin~T.~Montet; \twitter{benmontet}}
\email{b.montet@unsw.edu.au}

%@arxiver{f8a.pdf,f5.pdf,f2b.pdf}
%\date{\today, \currenttime}

\begin{abstract}

\btmtodo{Abstract goes here}
\end{abstract}

\keywords{Starspots (1572)}


 
\section{Introduction} \label{sec:intro}

As we've discovered more planetary systems, questions about formation, dynamical history, and evolution.

Many of these questions can begin to be answered through the understanding of the angular momentum of the system.

Commonly this is done through measurements of the Rossiter-McLaughlin effect to compare the spin axis of the star to the rotational axis of the planet.

Spins of companions rarely known. (See Mazeh+ for statistical discussion). Spins will be important for understanding evolution (Sarah's paper on this with Greg L, look also at their introduction). Look at discussion from Schwartz et al. 2016. Don't forget Josh + Josh 2009 paper on HD 189 which was the first to provide constraints on this axis (heh). Biersteker \& Schlichting talk about this in the context of depth variations.

But enter VHS 1256. Edge-on (albeit with large error bars). Orbits a binary. Astometric monitoring underway.

Was observed with TESS, what can TESS say about rotation periods?

First system with a bound substellar companion where spin-alignment of host star(s) and planet can be undertaken? As orbit is monitored, this will become a benchmark system to measure just about every angle you want in this system.

The rest of this paper is organized as follows.


\section{Data}

\subsection{Photometry}

\subsubsection{TESS}

The Transiting Exoplanet Survey Satellite \citep[\tess][]{Ricker15} observes $\approx 2300$ deg$^2$ regions of the sky for approximately 25-day intervals. 
Data for the entire field of view is obtained at 30-minute cadence, with $\sim 10,000$ pre-selected targets recorded at two-minute cadence.

\thisstar was observed in Sector 10 of the \tess\ mission as TIC 2470992 in both observing modes.



\subsubsection{HST}


\subsection{Spectroscopy}

\subsubsection{VLT}

\subsubsection{IGRINS}


\section{Results}

\subsection{Measuring Rotation Periods}

\btmtodo{Put analysis here}

\btmtodo{Mention beat frequency}

\citet{Zhou20} obtained \anytodo{Number! Close to 36?} hours of photometric data of the combined light of this system with the Spitzer Space Telescope. 
These data were obtained as three sequential Astronomical Observation Requests (AORs) and were treated independently in the removal of instrumental systematics from this dataset. 
In that analysis, \citet{Zhou20} detected a signal in one of their AORs with a period of $127 \pm 2$ minutes and an amplitude of $0.074 \pm 0.007\%$.
They were unable to determine if this signal was astronomical or instrumental in origin.
These observations are consistent with the rotation period inferred from \tess\ and \hst. 

Based on the two periods inferred from \tess, a 36-hour baseline should provide data covering approximately one half of the period of a beat cycle, so detecting the signal in only a fraction of the available data, when the two signals are in phase, is sensible. Moreover, the amplitude \btmtodo{blah blah blah is consistent...or is it?}



\subsection{Measuring Spin Axes}

This is hard! But we have people who are good at it.


\subsection{The Inclinations}


\section{Discussion}

Angular momentum in the system is all aligned, what does this imply?

The two host stars are rotating very quickly (even for young M dwarfs). The BD is rotating much more slowly (and also slowly for a BD?) What does this imply about formation? Can we argue that accretion processes probably played a significant role rather than a gravitational collapse?

\anytodo{what else?}


\section{Future Work and Conclusions}

Astrometric monitoring! We can know everything.

TESS will give more data in Sector 37 (Observed ~May 2021), better understand variability. 










%\begin{figure}[!tbh]
%  \begin{center}
%    \includegraphics[width=0.5\textwidth, trim={0cm 0.0cm 0cm 0cm}, clip=true]{example-image}
%   \end{center}
%  \caption{Test figure}
%  \label{fig:data}
%\end{figure}





\acknowledgements

We thank \btmtodo{people} for \btmtodo{things}.






This paper includes data collected by the \tess\ mission. Funding for the \tess\ mission is provided by the NASA Explorer Program.

\tess\ data were obtained from the Mikulski Archive for Space Telescopes
(MAST).
STScI is operated by the Association of Universities for Research in
Astronomy, Inc., under NASA contract NAS5-26555.
Support for MAST is provided by the NASA Office of Space Science via grant
NNX13AC07G and by other grants and contracts.



\software{%
    numpy \citep{numpy},
    matplotlib \citep{matplotlib},
    scipy \citep{Jones01}
    astropy \citep{Astropy18},
    eleanor \citep{Feinstein19},
    starry \citep{Luger19},
    emcee \citep{Foreman-Mackey12}
    }

\facilities{TESS}




\bibliography{exopapers}






\end{document}

